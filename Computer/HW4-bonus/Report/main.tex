\title{گزارش تمرین چهارم پردازش سیگنال‌های گرافی (امتیازی)}
\author{محمد رضیئی فیجانی}
%\date{15 بهمن 1400}


\makeatletter
\csvset{
	myautotabularcenter/.style={
		file=#1,
		after head=\csv@pretable\begin{tabular}{|>{\columncolor{lightgray}}c|*{\csv@columncount}{c|}}\csv@tablehead,
			table head=\hline مرکز خوشه & \csvlinetotablerow\\\hline,
			before line=  & \csvlinetotablerow,
			before first line= اعضای خوشه & \csvlinetotablerow,
			command=  & \csvlinetotablerow,
			late after line=\\,
			table foot=\\\hline,
			late after last line=\csv@tablefoot\end{tabular}\csv@posttable,
	},
}
\makeatother


\begin{document}
\maketitle
%\begin{latin}\lstinputlisting{codes/scanReduceIdxKernel.cu}\end{latin}


\section{بخش اول}

\subsection{}

نحوه بارگزاری داده و  محاسبه‌ی $Z$ :

\begin{latin}
\begin{lstlisting}
load('Data.mat')
Z = gsp_distanz(data').^2;
\end{lstlisting}
\end{latin}

\subsection{}

نحوه آموزش با تابع
\verb|gsp_learn_graph_log_degrees|

\begin{latin}
\begin{lstlisting}
a = 5; b = 5;
[W1] = gsp_learn_graph_log_degrees(Z, a, b);
W1(W1<1e-3) = 0; % clean up zeros
G1 = gsp_graph(W1);
draw_animal_graph(G1.L,names);
\end{lstlisting}
\end{latin}


نحوه آموزش با تابع
\verb|gsp_learn_graph_l2_degrees|

\begin{latin}
\begin{lstlisting}
a = 5;
[W2] = gsp_learn_graph_l2_degrees(Z, a);
W2(W2<1e-3) = 0; % clean up zeros
G2 = gsp_graph(W2);
draw_animal_graph(G2.L,names);
\end{lstlisting}
\end{latin}

خروجی های این دو گراف به شرح زیر می‌باشند.
\begin{figure}[h!]
	\centering
	\subfigure[\lr{W1}]{
		\includegraphics[width=0.45\linewidth]{figs/test-gsp-learn-graph-log-degrees-Z-5-5}
		\label{fig:test-gsp-learn-graph-log-degrees-z-5-5}}
	\hfill
	\subfigure[\lr{W2}]{
		\includegraphics[width=0.45\linewidth]{figs/test-gsp-learn-graph-12-degrees-Z-5}
		\label{fig:test-gsp-learn-graph-12-degrees-z-5}}
	\caption{}
\end{figure}
\FloatBarrier




\subsection{}


برای بررسی تاثیر پارامتر های $a,b$ در تابع 
\verb|gsp_learn_graph_log_degrees| 
از تکه کد زیر استفاده کردیم.

\begin{latin}
\begin{lstlisting}
a = 5; b = 5;
[W] = gsp_learn_graph_log_degrees(Z, a, b);
if max(max(W)) > 1e-3,  W = W / max(max(W)); end % nomalized by max value
W(W<1e-3) = 0; % clean up zeros
G = gsp_update_weights(G1,W);
draw_animal_graph(G.L,names);
text(0.3,-0.05,sprintf('gsp-learn-graph-log-degrees(Z, %g, %g)', a, b), "Units", "normalized")
\end{lstlisting}
\end{latin}

به ازای ثابت نگه داشتن یکی از دو پارامتر $a$ و $b$ و تغییر دیگری، نتیجه زیر حاصل شده است.


\begin{center}
	%	\begin{longtable}{>{\raggedleft\arraybackslash}p{0.18\linewidth}lp{0.4\linewidth}}
	%	\def\imagewidthincomptable{0.2\linewidth}
%	\def\nextline{\newline\vspace{-0.1em}}
%	\newcommand\addCompImage[1]{\raisebox{-0.89\totalheight}{\adjincludegraphics[width=\linewidth, trim={{0.15\width} {0.1\height} {0.12\width} {0.1\height}},clip]{#1}}\vspace{0.3em}}
%	\newcommand\addCompImage[1]{\raisebox{-0.89\totalheight}{\adjincludegraphics[width=\linewidth]{#1}}\vspace{0}}
%	\footnotesize
	\begin{longtable}{ccC{0.65\linewidth}}
		\caption{مقایسه نتایج با تغییر پارامتر $a,b$} \label{tab:comparison-a-b} \vspace{-1em} \\
		
		
		\hline \multicolumn{1}{c}{\textbf{ردیف}} &  \multicolumn{1}{c}{\textbf{مشخصات}} & \multicolumn{1}{c}{\textbf{گراف خروجی}} \\\hline 
		\endfirsthead
		
		\multicolumn{3}{c}%
		{{\bfseries \tablename\ \thetable{}: ادامه‌ی جدول مقایسه نتایج با تغییر پارامتر $a,b$}} \\
		\hline \multicolumn{1}{c}{\textbf{ردیف}} &  \multicolumn{1}{c}{\textbf{مشخصات}} & \multicolumn{1}{c}{\textbf{گراف خروجی}} \\\hline 
		\endhead
		
		\hline \multicolumn{3}{l}{{ادامه در صفحه بعد}} \\ \hline
		\endfoot
		
		\hline \hline
		\endlastfoot
		
		\rownumber & 
		$a = 5, b = 5$ & \raisebox{-0.98\totalheight}{\includegraphics[width=\linewidth]{figs/gsp-learn-graph-log-degrees-Z-5-5}} \\\hline
		
		\rownumber & 
		$a = 5, b = 100$ & \raisebox{-0.98\totalheight}{\includegraphics[width=\linewidth]{figs/gsp-learn-graph-log-degrees-Z-5-100}} \\\hline

		\rownumber & 
		$a = 5, b = 10$ & \raisebox{-0.98\totalheight}{\includegraphics[width=\linewidth]{figs/gsp-learn-graph-log-degrees-Z-5-10}} \\\hline
				
		\rownumber & 
		$a = 5, b = 1$ & \raisebox{-0.98\totalheight}{\includegraphics[width=\linewidth]{figs/gsp-learn-graph-log-degrees-Z-5-1}} \\\hline		
		\rownumber & 
		$a = 5, b = 0.1$ & \raisebox{-0.98\totalheight}{\includegraphics[width=\linewidth]{figs/gsp-learn-graph-log-degrees-Z-5-0.1.png}} \\\hline
		
		\rownumber & 
		$a = 5, b = 0$ & \raisebox{-0.98\totalheight}{\includegraphics[width=\linewidth]{figs/gsp-learn-graph-log-degrees-Z-5-0}} \\\hline
		
		\rownumber & 
		$a = 1000, b = 5$ & \raisebox{-0.98\totalheight}{\includegraphics[width=\linewidth]{figs/gsp-learn-graph-log-degrees-Z-1000-5}} \\\hline		
		\rownumber & 
		$a = 100, b = 5$ & \raisebox{-0.98\totalheight}{\includegraphics[width=\linewidth]{figs/gsp-learn-graph-log-degrees-Z-100-5}} \\\hline
		
		\rownumber & 
		$a = 10, b = 5$ & \raisebox{-0.98\totalheight}{\includegraphics[width=\linewidth]{figs/gsp-learn-graph-log-degrees-Z-10-5}} \\\hline
		
		\rownumber & 
		$a = 1, b = 5$ & \raisebox{-0.98\totalheight}{\includegraphics[width=\linewidth]{figs/gsp-learn-graph-log-degrees-Z-1-5}} \\\hline
		
		\rownumber & 
		$a = 0.1, b = 5$ & \raisebox{-0.98\totalheight}{\includegraphics[width=\linewidth]{figs/gsp-learn-graph-log-degrees-Z-0.1-5.png}} \\\hline		
			
\end{longtable}
\end{center}

این دو پارامتر در حقیقت اسپارسیتی گراف خروجی را مشخص می‌کنند. اما آن جنان معنادار نیستند.
مقاله‌ی \cite{largescale-kalofolias2019large}، 
این دو پارامتر را به دو پارامتر $\gamma,\theta$ می‌نگارد که $\theta$ را می‌توان جهت رسیدن به اسپارستی مطلوب تنظیم کرد. اگر هر رأس به طور میانگین با $k$ مجاور باشد، می‌توان با استفاده از تابع 
\verb|gsp_compute_graph_learning_theta(Z,k)|
پارامتر جدید $\theta$ را تعیین کرد که به شکل زیر از آن استفاده می‌شود.

\begin{latin}
\begin{lstlisting}
k = 8;
theta = gsp_compute_graph_learning_theta(Z, k); 
[W] = gsp_learn_graph_log_degrees(theta * Z, 1, 1);
if max(max(W)) > 1e-3,  W = W / max(max(W)); end
W(W<1e-3) = 0; % clean up zeros
G = gsp_update_weights(G1,W);
draw_animal_graph(G.L,names);
text(0.3,-0.05,sprintf('gsp-learn-graph-log-degrees(%g Z,1,1)', theta), "Units", "normalized")
\end{lstlisting}
\end{latin}

در این صورت می‌توان با تغییر $k$، نتایج معنا دار تری را گرفت که در جدول زیر، این نتایج آورده شده‌اند.


\begin{center}
%	\footnotesize
	\begin{longtable}{ccC{0.65\linewidth}}
		\caption{مقایسه نتایج با تغییر پارامتر $k$} \label{tab:comparison-k} \vspace{-1em} \\
		
		
		\hline \multicolumn{1}{c}{\textbf{ردیف}} &  \multicolumn{1}{c}{\textbf{مشخصات}} & \multicolumn{1}{c}{\textbf{گراف خروجی}} \\\hline 
		\endfirsthead
		
		\multicolumn{3}{c}%
		{{\bfseries \tablename\ \thetable{}: ادامه‌ی جدول مقایسه نتایج با تغییر پارامتر $k$}} \\
		\hline \multicolumn{1}{c}{\textbf{ردیف}} &  \multicolumn{1}{c}{\textbf{مشخصات}} & \multicolumn{1}{c}{\textbf{گراف خروجی}} \\\hline 
		\endhead
		
		\hline \multicolumn{3}{l}{{ادامه در صفحه بعد}} \\ \hline
		\endfoot
		
		\hline \hline
		\endlastfoot
		
		\rownumber & 
		$k = 4$ & \raisebox{-0.98\totalheight}{\includegraphics[width=\linewidth]{figs/gsp-learn-graph-log-degrees-Z-k4}} \\\hline
				
		\rownumber & 
		$k = 5$ & \raisebox{-0.98\totalheight}{\includegraphics[width=\linewidth]{figs/gsp-learn-graph-log-degrees-Z-k5}} \\\hline
				
		\rownumber & 
		$k = 8$ & \raisebox{-0.98\totalheight}{\includegraphics[width=\linewidth]{figs/gsp-learn-graph-log-degrees-Z-k8}} \\\hline
				
		\rownumber & 
		$k = 10$ & \raisebox{-0.98\totalheight}{\includegraphics[width=\linewidth]{figs/gsp-learn-graph-log-degrees-Z-k10}} \\\hline
				
		\rownumber & 
		$k = 15$ & \raisebox{-0.98\totalheight}{\includegraphics[width=\linewidth]{figs/gsp-learn-graph-log-degrees-Z-k15}} \\\hline
						
		\rownumber & 
		$k = 20$ & \raisebox{-0.98\totalheight}{\includegraphics[width=\linewidth]{figs/gsp-learn-graph-log-degrees-Z-k20}} \\\hline
		
			
	\end{longtable}
\end{center}

\subsection{}%4

\subsection{}%5

\subsection{}%6

\section{بخش دوم}


\subsection{} %1

\begin{latin}
\begin{lstlisting}
function cs = cosine_similarity(x,y) % row data
if nargin < 2, y = x; end
n = size(x, 1);
m = size(y, 1);
normX = sqrt(sum(abs(x).^2, 2));
normY = sqrt(sum(abs(y).^2, 2));
cs = zeros(n,m);
for i = 1:n
	for j = 1:m
		cs(i,j) = x(i,:) * y(j,:)' / (normX(i) * normY(j));
	end
end
end
\end{lstlisting}
\end{latin}


\begin{latin}
\begin{lstlisting}
function Gquery = query2graph(tokens, k, normalized)
if nargin < 2, k = 5; end
if nargin < 3, normalized = true; end
CS = cosine_similarity(tokens);
Z = zero_diag(1 - CS).^2; % cosine distance
theta = gsp_compute_graph_learning_theta(Z, k);
W = gsp_learn_graph_log_degrees(theta*Z, 1, 1);
W(W<1e-3) = 0; % clean up zeros
if normalized && max(W(:)) > 1e-3, W = W / max(W(:)); end
G = gsp_ring(size(W,1));
Gquery = gsp_graph(W, G.coords);
end
\end{lstlisting}
\end{latin}



\subsection{} %2

\begin{latin}
\begin{lstlisting}
queries_csv = readtable('queries.csv');
queries_csv.tokens = arrayfun(@(qt) jsondecode(strrep(qt,'''', '"')), queries_csv.tokens, 'UniformOutput', false);
num_queries = size(queries_csv, 1);
load('tokens_of_queries.mat')
load('queries.mat')
\end{lstlisting}
\end{latin}

\begin{LTR}
\begin{latin}
\begin{lstlisting}
Gqueries = cell(1, num_queries);
for qi = 1:num_queries
	tokens = tokens_of_queries{qi};
	Gqueries{qi} = query2graph(tokens, floor(0.35*size(tokens,1)));
	draw_animal_graph(Gqueries{qi}.L, queries_csv.tokens{qi});
	set(gcf, 'Name', sprintf('query: %g',qi), 'MenuBar', 'none', ...
	'numbertitle', 'off', 'Units', 'normalized', 'Position',[0.4 0.05 0.55 0.9]);
end
\end{lstlisting}
\end{latin}
\end{LTR}

نتایج آن در جدول زیر آورده شده است.


\newpage
\begin{center}
	\newcommand{\addQueryGraphImage}[1]{#1 & %
		\raisebox{-0.98\totalheight}{\includegraphics[width=\linewidth]{figs/partB-query2graph-#1}}\\}
	\begin{longtable}{cC{0.65\linewidth}}
		\caption{گراف نظیر کوئری های مختلف} \label{tab:graph2query} \vspace{-1em} \\
		
		\hline \multicolumn{1}{c}{\textbf{شماره کوئری}} & \multicolumn{1}{c}{\textbf{گراف خروجی}} \\\hline 
		\endfirsthead
		
		\multicolumn{2}{c}%
		{{\bfseries \tablename\ \thetable{}:ادامه جدول گراف نظیر کوئری های مختلف}} \\
		\hline \multicolumn{1}{c}{\textbf{شماره کوئری}} & \multicolumn{1}{c}{\textbf{گراف خروجی}} \\\hline 
		\endhead
		
		\hline \multicolumn{2}{l}{{ادامه در صفحه بعد}} \\ \hline
		\endfoot
		
		\hline \hline
		\endlastfoot
		
%		\makeatletter
%		\@whilenum\value{magicrownumbers}<10\do
%		{\stepcounter{magicrownumbers}\addQueryGraphImage{\themagicrownumbers}\hline}
%		\makeatother
		
		\addQueryGraphImage{1}\hline
		\addQueryGraphImage{2}\hline
		\addQueryGraphImage{3}\hline
		\addQueryGraphImage{4}\hline
		\addQueryGraphImage{5}\hline
		\addQueryGraphImage{6}\hline
		\addQueryGraphImage{7}\hline
		\addQueryGraphImage{8}\hline
		\addQueryGraphImage{9}\hline
		\addQueryGraphImage{10}\hline
		
	\end{longtable}
\end{center}




\subsection{} %3
با استفاده از \lr{Add-Ons} متلب نصب شد.

\subsection{} %4

\begin{latin}
\begin{lstlisting}
X = cell(1,num_queries);
for qi = 1:num_queries
	X{qi} = cosine_similarity(tokens_of_queries{qi}, queries{qi});
	VV = GCModulMax1(Gqueries{qi}.W);
	VVmax = max(VV);
	Vmask = VV == unique(VV)';
	Vmax = max(sum(Vmask,1));
	[~,I] = max(X{qi}.*Vmask,[],1);
	tokens = queries_csv.tokens{qi};
	clusters = cell(Vmax+1, VVmax);
	for v = 1:VVmax
		mask = Vmask(:,v);
		clusters(:,v) = [tokens(I(v)); tokens(mask); repmat({' '}, Vmax - sum(mask),1)];
	end
	writetable(cell2table(clusters(2:end,:),'VariableNames',clusters(1,:)), sprintf('Report/figs/partB-clusters-%g.csv',qi))
end
\end{lstlisting}
\end{latin}


کوئری 1:
\newline\csvreader[myautotabularcenter]{figs/partB-clusters-1.csv}{}

\newline\newline کوئری 2:
\newline\csvreader[myautotabularcenter]{figs/partB-clusters-2.csv}{}

\newline\newline کوئری 3:
\newline\csvreader[myautotabularcenter]{figs/partB-clusters-3.csv}{}

\newline\newline کوئری 4:
\newline\csvreader[myautotabularcenter]{figs/partB-clusters-4.csv}{}

\newline\newline کوئری 5:
\newline\csvreader[myautotabularcenter]{figs/partB-clusters-5.csv}{}

\newline\newline کوئری 6:
\newline\csvreader[myautotabularcenter]{figs/partB-clusters-6.csv}{}

\newline\newline کوئری 7:
\newline\csvreader[myautotabularcenter]{figs/partB-clusters-7.csv}{}

\newline\newline کوئری 8:
\newline\csvreader[myautotabularcenter]{figs/partB-clusters-8.csv}{}

\newline\newline کوئری 9:
\newline\csvreader[myautotabularcenter]{figs/partB-clusters-9.csv}{}

\newline\newline کوئری 10:
\newline\csvreader[myautotabularcenter]{figs/partB-clusters-10.csv}{}








\subsection{} %5







\section{بخش سوم}


\subsection{} %1
نحوه خواندن اطلاعات در این قسمت مانند قسمت قبل است.

\begin{latin}
\begin{lstlisting}
queries_csv = readtable("queries.csv");
queries_csv.tokens = arrayfun(@(qt) jsondecode(strrep(qt,'''', '"')), queries_csv.tokens, 'UniformOutput', false);
num_queries = size(queries_csv, 1);
load('tokens_of_queries.mat');
load('queries.mat');
load('docs.mat'); docs = cell2mat(docs);
\end{lstlisting}
\end{latin}



\subsection{} %2


\begin{figure}[h!]
	\centering
	\includegraphics[width=0.7\linewidth]{figs/partC-Gdocs-W}
	\caption{}
	\label{fig:partc-gdocs-w}
\end{figure}\FloatBarrier




\subsection{} %3

\begin{latin}
\begin{lstlisting}
X = cosine_similarity(docs, queries{1});
[Gs, ms, ys, xs] = MyPyramidAnalysis(Gdocs, X, 7);
Gselected = Gs{end};
N = Gselected.N; % 17
Xselected = xs{end};
Mselected = ms{end};
Xsampled = X(Mselected);
disp((Xsampled/max(X))') 
figure('Color','w'); gsp_plot_signal(Gdocs, X);
figure('Color','w'); gsp_plot_signal(Gselected, Xselected);
\end{lstlisting}
\end{latin}




\begin{figure}[h!]
	\centering
	\subfigure[]{
		\includegraphics[width=0.45\linewidth]{figs/partC-Gdocs}
		\label{fig:partc-gdocs}}
	\hfill
	\subfigure[]{
		\includegraphics[width=0.45\linewidth]{figs/partC-Gselected}
		\label{fig:partc-gselected}}
	\caption{}
\end{figure}\FloatBarrier


\begin{figure}[h!]
	\centering
	\includegraphics[width=0.7\linewidth]{figs/partC-Gsampled}
	\caption{}
	\label{fig:partc-gsampled}
\end{figure}\FloatBarrier








%%%%%%%%%%%%%%%%%%%%%%%%%%%%%%%%%%%%%%%%%%%%%%%%%%%%%%%%%%%%%%%%%%%%%%%%%%%%%%%%%%%%%%%%
\newpage
\begin{latin}
	\baselineskip=.8\baselineskip
	%\bibliographystyle{./styles/packages/unsrtabbrv}
	\bibliographystyle{plainurl}
	
	% Uncomment next line to include uncited references
	\nocite{*}
	\bibliography{refs}
\end{latin}

\end{document}